\documentclass[UTF8]{report}
\usepackage{ctex}

\begin{document}
    \section*{18.}
        \subsection*{(a)}
            $$E[X] = \int_{1}^{3}\frac{x^2}{4}dx = \frac{13}{6}$$
            $$P(A) = 1 - P(X \leq 2) = 1 - \int_{1}^{2}\frac{x}{4}dx = \frac{5}{8}$$
            $$f_{X|A}(x) = \left\{
                \begin{array}{lcr}
                    \frac{2x}{5} & & 2 \leq x \leq 3\\
                    0 & & otherwise
                \end{array}
            \right.$$
            $$E[X|A] = \int_{2}^{3}\frac{2x^2}{5}dx = \frac{38}{15}$$
        \subsection*{(b)}
            $$E[Y] = E[X^2] = \int_{1}^{3}\frac{x^3}{4}dx = 5$$
            $$var(Y) = E[Y^2] - E[Y]^2 = \int_{1}^{3}\frac{x^5}{4}dx - 25 = \frac{16}{3}$$
    \section*{19.}
        \subsection*{(a)}
            由归一性:$\int_{1}^{2}cx^{-2}dx = 1$,解得$c = 2$。
        \subsection*{(b)}
            $$P(A) = 1 - P(X \leq 1.5) = 1 - \int_{1}^{1.5}\frac{2dx}{x^2} = \frac{1}{3}$$
            $$f_{X|A}(x) = \left\{
                \begin{array}{lcr}
                    \frac{6}{x^2} & & 1.5 \leq x \leq 2\\
                    0 & & otherwise
                \end{array}
            \right.$$
        \subsection*{(c)}
            $$E[Y|A] = E[X^2|A] = \int_{1.5}^{2}6dx = 3$$
            $$var(Y|A) = E[Y^2|A] - E[Y|A]^2 = \frac{1}{4}$$
    \section*{20.}
        因为共同的分布为指数分布,且期望值为30,所以$f_X(x) = \frac{1}{30}e^{-\frac{x}{30}}$。
        第二个学生到达时,第一个学生已经到达了5分钟,所以$P(X \geq 5) = 1 - \int_{0}^{5}f_X(x) = e^{-\frac{1}{6}}$。
        所以第一个学生答疑时间的概率密度函数变为$f_{X|A}(x) = \frac{e^{\frac{1}{6}}}{30}e^{-\frac{x}{30}}$,
        其期望值为$E[X|A] = \int_{5}^{\infty}\frac{e^{\frac{1}{6}}x}{30}e^{-\frac{x}{30}}dx = 35$,所以所花时间的期望值为65。
    \section*{21.}
        \subsection*{(a)}
            $$f_{X, Y}(x, y) = \left\{
                \begin{array}{lcr}
                    \frac{1}{xl} & & y \leq x \leq l\\
                    0 & & otherwise
                \end{array}
            \right.$$
        \subsection*{(b)}
            $$f_{Y}(y) = \left\{
                \begin{array}{lcr}
                    \int_{y}^{l}\frac{dx}{xl} & & 0 \leq y \leq l\\
                    0 & & otherwise
                \end{array}
            \right.$$
            
            解之得:
            $$f_{Y}(y) = \left\{
                \begin{array}{lcr}
                    \frac{lnl - lny}{l} & & 0 \leq y \leq l\\
                    0 & & otherwise
                \end{array}
            \right.$$
        \subsection*{(c)}
            $$E[Y] = \int_{0}^{l}\frac{y(lnl - lny)}{l}dy = \frac{l}{4}$$
        \subsection*{(d)}
            $E[Y] = E[X\frac{Y}{X}] = E[X]E[\frac{Y}{X}] = \frac{l}{2}\frac{1}{2} = \frac{l}{4}$。
    \section*{22.}
        \subsection*{(i)}
            设随机选取的两个点距离左侧的距离为X和Y,不妨认为X为更靠右侧的那个点,则概率密度函数为:
            $$f_{X, Y}(x, y) = \left\{
                \begin{array}{lcr}
                    \frac{1}{x} & & 0 \leq y \leq x \leq 1\\
                    0 & & otherwise
                \end{array}
            \right.$$

            由于三角形的两边之和大于第三边,我们可以列出满足条件的三个式子:
            $$\begin{array}{lcr}
                Y + X - Y & > & 1 - X\\
                Y + 1 - X & > & X - Y\\
                X - Y + 1 - X & > & Y\\
            \end{array}$$
            
            整理后得到:
            $$\begin{array}{lccrr}
                X & > & \frac{1}{2} & \rightarrow & A\\
                Y & < & \frac{1}{2} & \rightarrow & B\\
                X - Y & < & \frac{1}{2} & \rightarrow & C\\
            \end{array}$$

            所以$P(A) = 1 - \int_{0}^{\frac{1}{2}}\int_{0}^{x}\frac{dydx}{x} = \frac{1}{2}$,
            $P(B|A) = \int_{\frac{1}{2}}^{1}\int_{0}^{\frac{1}{2}}\frac{2}{x}dydx = ln2$,
            $P(C|(B|A)) = \int_{\frac{1}{2}}^{1}\int_{x - \frac{1}{2}}^{\frac{1}{2}}\frac{dydx}{xln2} = 1 - \frac{1}{2ln2}$。
        \subsection*{(ii)}
            本质上和(i)是一样的,仅仅只是选取方向的不同。
        \subsection*{(iii)}
            设Y为第一次选取的点距离最近的一端的距离,X为第二次选区的点距离Y的距离,则概率密度函数为:
            $$f_{X, Y}(x, y) = \left\{
                \begin{array}{lcr}
                    \frac{2}{1 - y} & & 0 \leq y \leq \frac{1}{2}, y \leq x \leq 1\\
                    0 & & otherwise
                \end{array}
            \right.$$

            同(i),我们可以列出以下的条件(注意,此时的三段杆的距离已经变成了$X, Y, 1 - X - y$):
            $$\begin{array}{lccrr}
                X & < & \frac{1}{2} & \rightarrow & A\\
                Y & < & \frac{1}{2} & \rightarrow & B\\
                X + Y & > & \frac{1}{2} & \rightarrow & C\\
            \end{array}$$

            所以$P(A) = \int_{0}^{\frac{1}{2}}\int_{y}^{\frac{1}{2}}\frac{2}{1 - y}dxdy = 1 - ln2$,
            $P(B|A) = 1$,$P(C|(B|A)) = 2ln\frac{4}{3} + \frac{1}{2}ln\frac{3}{2} - \frac{1}{4}$。
    \section*{23.}
        \subsection*{(a)}
            $$f_{X, Y}(x, y) = \left\{
                \begin{array}{lcr}
                    2 & & x \geq 0, y \geq 0, x + y \leq 1\\
                    0 & & otherwise
                \end{array}
            \right.$$
        \subsection*{(b)}
            $$f_Y(y) = \int_{0}^{1 - y}2dx = 2(1 - y)$$
        \subsection*{(c)}
            $$f_{X|Y = y}(x) = \frac{1}{1 - y}$$
        \subsection*{(d)}
            $$E[X|Y = y] = \frac{1 - y}{2}$$
            $$E[X] = \int_{0}^{1}E[X|Y = y]F_{Y}(y)dy = \int_{0}^{1}(1 - y)^2dy = \frac{1}{3}$$
        \subsection*{(e)}
            利用对称性,$E[Y] = E[X] = \frac{1}{3}$。
    \section*{24.}
        概率密度函数:
        $$f_{X, Y}(x, y) = \left\{
            \begin{array}{lcr}
                1 & & x \geq 0, y \geq 0, 2x + y \leq 2\\
                0 & & otherwise
            \end{array}
        \right.$$

        边缘概率密度函数:
        $$f_X(x) = \int_{0}^{-2x + 2}dy = -2x + 2$$
        $$f_Y(y) = \int_{0}^{\frac{2 - y}{2}}dx = \frac{2 - y}{2}$$

        条件期望:
        $$E[X|Y = y] = \frac{2 - y}{4}$$
        $$E[Y|X = x] = 1 - x$$

        所以由全期望定理可得:
        $$E[X] = \int_{0}^{1}E[X|Y = y]f_Y(y)dy = \frac{5}{24}$$
        $$E[Y] = \int_{0}^{2}E[Y|X = x]f_X(x)dx = \frac{4}{3}$$
    \section*{25.}
        $(X, Y)$离原点的距离至少为c,故$\sqrt{X^2 + Y^2} \geq c$,
        所以$P(A) =\\ P(\sqrt{X^2 + Y^2} \geq c) = 1 - \int_{0}^{2\pi}\int_{0}^{c}\frac{\rho}{2\pi \sigma^2}e^{\frac{-\rho^2}{2\sigma^2}}d\rho d\theta = e^{\frac{-c^2}{2\sigma^2}}$。

        于是得到条件联合概率密度函数:
        $$f_{(X, Y)|A} = \frac{1}{2\pi \sigma^2}e^{\frac{-x^2 - y^2 + c^2}{2\sigma^2}}$$
    \section*{26.}
        略。
    \section*{27.}
        \subsection*{(a)}
            $$\int_{(x, y) \in A}f_{X,Y|C}(x, y)dxdy = \frac{1}{P(C)}\int_{(x, y) \in A}f_{X, Y}(x, y)dxdy = \frac{P(C)}{P(C)} = 1$$
            所以$f_{X,Y|C}(x, y)$是一个合格的联合概率密度函数。
        \subsection*{(b)}
            令$C = \bigcup_{i = 1}^{n}C_i$,因为$A_i$是二维平面的分割,所以$C_i$也是$C$的一个分割。
            由(a)可知:$f_{X, Y}(x, y) = P(C)f_{X, Y|C}(x, y)$,所以由全概率公式即可得到
            $f_{X, Y}(x, y) = \sum_{i = 1}^{n}P(C_i)f_{X, Y|C_i}(x, y)$。
    \section*{28.}
        略。
    \section*{29.}
        略。
    \section*{30.}
        略。
    \section*{31.}
        略。
    \section*{32.}
        略。
    \section*{33.}
        略。
\end{document}