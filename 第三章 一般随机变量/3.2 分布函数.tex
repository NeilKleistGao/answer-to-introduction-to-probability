\documentclass[UTF8]{report}
\usepackage{ctex}

\begin{document}
    \section*{5.}
        不妨设三角形的高为H,对应的底边长度为L,
        则分布函数$P_X(x)$可以用梯形的面积除以三角形面积得到:
        $P_X(x) = \frac{\frac{H - x}{H}L + L}{2}\frac{2}{HL} = \frac{x(2H - x)}{H^2}, x \in [0, H]$。

        所以概率密度函数$f_X(x) = P_X'(x) = \frac{2(H - x)}{H^2}$。
    \section*{6.}
        有0个顾客的时候,等候的时间为0;有一个顾客的时候,等候时间是一个指数随机变量。
        所以:
        $$P_X(x) = \left\{
            \begin{array}{lcr}
                0 & & x < 0\\
                \frac{1}{2} & & x = 0\\
                \frac{1}{2}(1 + \int_{0}^{x}\lambda e^{-\lambda t}dt) & & x > 0
            \end{array}
        \right.$$

        解之得:
        $$P_X(x) = \left\{
            \begin{array}{lcr}
                0 & & x < 0\\
                \frac{1}{2} & & x = 0\\
                \frac{1}{2}(2 - e^{-\lambda x}) & & x > 0
            \end{array}
        \right.$$
    \section*{7.}
        \subsection*{(a)}
            X的分布函数为两个圆形面积的比值,所以$P_X(x) = \frac{x^2}{r^2}, x \in [0, r]$。
            求导后可以得到概率密度函数$f_X(x) = \frac{2x}{r^2}$。所以$E[X] = \int_{0}^{r}\frac{2x^2}{r^2}dx = \frac{2r}{3}$,
            $var(X) = E[X^2] - E[X]^2 = \frac{r^2}{18}$。
        \subsection*{(b)}
            由上一问可知:
            $$P_S(s) = \left\{
                \begin{array}{lcr}
                    0 & & s < 0\\
                    \frac{r^2 - t^2}{r^2} & & s = 0\\
                    \frac{r^2 - t^2}{r^2} & & 0 < s < \frac{1}{t}\\
                    \frac{r^2 - t^2}{r^2} + \frac{t^2 - (\frac{1}{s})^2}{r^2} & & s \geq \frac{1}{t}
                \end{array}
            \right.$$

            是连续随机变量。
    \section*{8.}
        \subsection*{(a)}
            X的分布函数为$P_X(X \leq x)$,由全概率公式,可以得到$P_X(X \leq x) = pP_Y(Y \leq x) + (1 - p)P_Z(Z \leq x)$。
            对式子两侧分别求导可得$f_X(x) = pf_Y(x) + (1 - p)f_Z(x)$。

            证毕。
        \subsection*{(b)}
            $$P_X(x) = \left\{
                \begin{array}{lcr}
                    pe^{\lambda x} & & x < 0\\
                    1 + (p - 1)e^{-\lambda x} & & x \geq 0
                \end{array}
            \right.$$
    \section*{9.}
        略。
    \section*{10.}
        略。
\end{document}