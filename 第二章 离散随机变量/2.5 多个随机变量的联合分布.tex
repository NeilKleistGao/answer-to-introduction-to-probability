\documentclass[UTF8]{report}
\usepackage{ctex}

\begin{document}
    \section*{24.}
        \subsection*{(a)}
            由题意得:X,Y是$-2 \leq x \leq 4, x - 1 \leq y \leq x + 1$上的均匀分布,所以:
            \begin{table}[!htbp]
                \begin{tabular}{llllllll}
                    Y/X & -2 & -1 & 0 & 1 & 2 & 3 & 4 \\
                    -3 & $\frac{1}{21}$ & 0 & 0 & 0 & 0 & 0 & 0\\
                    -2 & $\frac{1}{21}$ & $\frac{1}{21}$ & 0 & 0 & 0 & 0 & 0\\
                    -1 & $\frac{1}{21}$ & $\frac{1}{21}$ & $\frac{1}{21}$ & 0 & 0 & 0 & 0\\
                    0 & 0 & $\frac{1}{21}$ & $\frac{1}{21}$ & $\frac{1}{21}$ & 0 & 0 & 0\\
                    1 & 0 & 0 & $\frac{1}{21}$ & $\frac{1}{21}$ & $\frac{1}{21}$ & 0 & 0\\
                    2 & 0 & 0 & 0 & $\frac{1}{21}$ & $\frac{1}{21}$ & $\frac{1}{21}$ & 0\\
                    3 & 0 & 0 & 0 & 0 & $\frac{1}{21}$ & $\frac{1}{21}$ & $\frac{1}{21}$\\
                    4 & 0 & 0 & 0 & 0 & 0 & $\frac{1}{21}$ & $\frac{1}{21}$\\
                    5 & 0 & 0 & 0 & 0 & 0 & 0 & $\frac{1}{21}$\\
                \end{tabular}
            \end{table}

            所以边缘分布列$$P_X(x) = \left\{\begin{array}{lcr}
                \frac{1}{7} & & -2 \leq x \leq 4 \\
                0 & & otherwise
            \end{array}\right.$$

            $$P_Y(y) = \left\{\begin{array}{lcr}
                \frac{1}{21} & & y = -3/y = 5\\
                \frac{2}{21} & & y = -2/y = 4\\
                \frac{1}{7} & & -1 \leq y \leq 3\\
                0 & & otherwise
            \end{array}\right.$$

            期望$E[X] = 1, E[Y] = 1$。
        \subsection*{(b)}
            $$E[100X + 200Y] = \sum_{(x, y)}(100x + 200y)P_{X, Y}(x, y) = \frac{8200}{21} \approx 390$$
    \section*{25.}
        \subsection*{(a)}
            因为答案被选中的可能性相等,故$P_{I, J}$满足均匀分布,所以
            $$P_{I, J}(i, j) = \left\{
                \begin{array}{lcr}
                    \frac{1}{\sum_{k = 1}^{n}m_k} & & 0 < j \leq m_i\\
                    0 & & otherwise
                \end{array}
                \right.$$

            不难得到边缘分布列$P_I(i) = \frac{m_i}{\sum_{k = 1}^{n}m_k}, P_J(j) = \frac{\sum_{k = 1}^{n}[j \leq m_k]}{\sum_{k = 1}^{n}m_k}$。
        \subsection*{(b)}
            对于第i个学生,可以将其回答情况视为$m_i$个独立的题目的得分期望的和,所以总分的期望值为$\sum_{k = 0}^{m_i}ap_{i, k} + b(1 - p_{i, k})$。
    \section*{26.}
        \subsection*{(a)}
            因为最低分不小于x的分布列$P(X_1 \geq x, X_2 \geq x, X_3 \geq x) = \Pi_{i = 1}^3\frac{111 - x}{10}$

            所以$P_X(x) = (\frac{111 - x}{10})^3 - (\frac{110 - x}{10})^3$
        \subsection*{(b)}
            三天的平均得分为$\frac{x_1 + x_2 + x_3}{3}$,而X的期望$E[X] = \sum_{i = 101}^{110}x((\frac{111 - x}{10})^3 - (\frac{110 - x}{10})^3)$。

            平均分的期望值为$105.5$,而X的期望为$103.025$。
    \section*{27.}
        略。
    \section*{28.}
        换一种方式陈述书上的答案,使得答案更加容易理解。

        我们如果按照原来的顺序答题,奖金的期望$E[L] = p_1v_1 + p_1p_2v_2 + \dots + p_1p_2\dots p_nv_n$。

        假定我们尝试着交换相邻两道题的顺序(交换任意两道题的顺序可以通过若干次相邻交换实现),
        则交换后的期望值变为$E[L'] = p_1v_1 + \dots + p_1p_2\dots p_{k + 1}v_{k + 1} + p_1p_2\dots p_kp_{k + 1}v_k + \dots + p_1p_2\dots p_nv_n$。
        将两个式子做差,可以得到$p_1p_2\dots p_{k - 1}(p_kv_k + p_kp_{k + 1}v_{k + 1} - p_{k + 1}v_{k + 1} - p_kp_{k + 1}v_k)$。括号外的式子一定为正,
        我们只看括号内:$p_kv_k(1 - p_{k + 1}) + p_{k + 1}v_{k + 1}(p_k - 1)$,提出一个$(1 - p_{k + 1})(1 - p_{k})$,得到
        $\frac{p_kv_k}{1 - p_k} - \frac{p_{k + 1}v_{k + 1}}{1 - p_{k + 1}}$。所以先选$\frac{p_iv_i}{1 - p_i}$是明智的。

        证毕。
    \section*{29.}
        略。
    \section*{30.}
        略。
\end{document}