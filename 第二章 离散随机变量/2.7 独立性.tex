\documentclass[UTF8]{report}
\usepackage{ctex}

\begin{document}
    \section*{38.}
        设每个路口遇到红灯的概率为p。
        \subsection*{(a)}
            $$P_X(x) = C_4^xp^x(1 - p)^{4 - x}$$
            $$E[X] = 4p$$
            $$var(X) = \sum_{i = 0}^4var(X_i) = 4p(1 - p)$$
        \subsection*{(b)}
            $$var(2X) = 16p(1 - p)$$
    \section*{39.}
        由独立性,有:
        $$E[X] = \sum_{i = 1}^{10}E[X_i] = 35$$
        $$var(X) = \sum_{i = 1}^{10}var(X_i) = \frac{175}{6}$$
    \section*{40.}
        由于论文之间的成绩完全独立,我们不妨在论文与论文之间插入隔板,以此区分不同成绩的两批论文。
        如果上交了k篇论文,每种评分的论文至少有一篇的概率就为$P(k) = \frac{5!C_{k - 1}^5}{(k + 1)^5}$
        (一共需要安插5个隔板,分为6个不同评分的区间,所有的可能安插的位置一共是$k + 1$个,而合法的安插位置只能在中间的$k - 1$个中选择,且不能重复,隔板自身的顺序有$5!$种)。

        其中$P(24) \approx 0.5$,$\lim_{k \to \infty} = 1$。
    \section*{41.}
        \subsection*{(a)}
            不难得到$P(x) = C_{250}^x0.02^x0.98^{250 - x}$,所以$E[X] = 50$。

            所以罚单数刚好等于50的概率为$P(50)$。
        \subsection*{(b)}
            利用泊松分布近似(a)中的结果:$P(50) \approx e^{-50}\frac{(50)^{50}}{50!} \approx 0.056$。
        \subsection*{(c)}
            $$E[0.5 \times 10X + 0.3 \times 20X + 0.2 \times 50X] = 1050$$
            $$var(0.5 \times 10X + 0.3 \times 20X + 0.2 \times 50X) = 2160.9$$
        \subsection*{(d)}
            因为$\sigma(X) \approx 2.21$,所以$p \in [0, 0.064]$。
    \section*{42.}
        \subsection*{(a)}
            因为$P_{X_i} = S$,所以$E[X_i] = S$,故$E[S_n] = \frac{nS}{n} = S$。

            对于单个点的方差$var(X_i) = S(1 - S)$是有限值,
            所以由独立性$\\\lim_{n \to \infty}var(S_n) = \frac{nS(1 - S)}{n^2} = 0$。

            证毕。
        \subsection*{(b)}
            $$S_n = \frac{(n - 1)S_{n - 1} + X_n}{n}$$
        \subsection*{(c)}
            由于单位正方形的内切圆的半径为$\frac{1}{2}$,所以
            $$\pi = 4E[S_n] = \frac{4}{n}\sum_{i = 1}^{10000}P_{X_i}$$
            只要计算落在圆形内部的点的数量即可确定$\pi$的值。

        \subsection*{(d)}
            类似于(c)。
    \section*{43.}
        略。
    \section*{44.}
        略。
    \section*{45.}
        略。
    \section*{46.}
        略。
\end{document}