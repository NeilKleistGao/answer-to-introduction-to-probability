\documentclass[UTF8]{report}
\usepackage{ctex}

\begin{document}
    \section*{1.}
        \begin{table}[!htbp]
            \begin{tabular}{llllll}
                得分 & 0 & 1 & 2 & 3 & 4 \\
                概率 & 0.18 & 0.27 & 0.34 & 0.14 & 0.07
            \end{tabular}
        \end{table}
    \section*{2.}
        假设有k个人与你的生日相同,由二项分布:
        $$P_X(k) = C_{500}^kp^k(1 - p)^{500 - k}$$
        其中$p = \frac{1}{365}$,则有人生日与你相同的概率为$1 - P_X(0) = 1 - (1 - \frac{1}{365})^{500} \approx 0.75$。

        若使用泊松分布逼近,$\lambda = np = \frac{500}{365}$,
        那么$1 - P_X(0) = 1 - e^{-\lambda}\frac{\lambda^0}{0!} \approx 0.75$。
    \section*{3.}
        \subsection*{(a)}
            设k为比赛连续平局的局数,则$P(k) = 0.3^k$,那么赢得比赛的概率为$0.4\sum_{i = 0}^9P(i) = 0.5714251972‬$。
        \subsection*{(b)}
            下棋的前9局数满足几何分布,所以$P_X(k) = (1 - p)^{k - 1}p = 0.3^{k - 1}0.7$,
            第10局后比赛一定结束,
            所以分布列为:
            \begin{table}[!htbp]
                \begin{tabular}{llllll}
                    局数 & 1 & 2 & 3 & 4 & 5 \\
                    概率 & 0.7 & 0.21 & 0.063 & 0.0189‬ &
                    0.00567‬
                \end{tabular}
            \end{table}

            \begin{table}[!htbp]
                \begin{tabular}{llllll}
                    局数 & 6 & 7 & 8 & 9 & 10 \\
                    概率 & 0.001701‬ & 0.0005103‬ &
                     0.00015309‬ & 0.000045927‬ & 0.000019683‬
                \end{tabular}
            \end{table}
    \section*{4.}
        \subsection*{(a)}
            在使用者的数量小于等于50的时候,调制解调器的数量分布满足二项分布;
            使用者数量大于50的时候,由于调制解调器数量不足,故一定是50个。

            所以分布列为:
            \begin{table}[!htbp]
                \begin{tabular}{lll}
                    人数 & $k \leq 50$ & $k > 50$ \\
                    概率 & $C_{1000}^k0.01^k0.99^{1000 - k}$ & $1 - \sum_{i = 0}^{50}C_{1000}^i0.01^i0.99^{1000 - i}$ 
                \end{tabular}
            \end{table}
        \subsection*{(b)}
            设使用者人数为k,由泊松分布有$P_X(k) = e^{-\lambda}\frac{\lambda^k}{k!} = e^{-10}\frac{10^k}{k!}$
        \subsection*{(c)}
            利用精确分布,$P = 1 - \sum_{i = 0}^{50}C_{1000}^i0.01^i0.99^{1000 - i}$,而利用泊松分布,$P = 1 - \sum_{i = 0}^{50}e^{-10}\frac{10^i}{i!}$
    \section*{5.}
        \subsection*{(a)}
            第一时段结束时:
            \begin{table}[!htbp]
                \begin{tabular}{lll}
                    信息包数量k & $k < b$ & $k = b$ \\
                    概率 & $e^{-\lambda}\frac{\lambda^k}{k!}$ & $1 - \sum_{i = 0}^{b - 1}e^{-\lambda}\frac{\lambda^i}{i!}$
                \end{tabular}
            \end{table}

            第二时段结束时:
            \begin{table}[!htbp]
                \begin{tabular}{lll}
                    信息包数量k & $k = 0$ & $k > 0$ \\
                    概率 & $\sum_{i = 0}^c e^{-\lambda}\frac{\lambda^i}{i!}$ &
                    $\sum_{i = c + 1}^{b - 1} e^{-\lambda}\frac{\lambda^i}{i!} + 1 - \sum_{i = 0}^{b - 1}e^{-\lambda}\frac{\lambda^i}{i!}$
                \end{tabular}
            \end{table}
        \subsection*{(b)}
            如果到达的信息包的数量不超过b,是不会发生丢包的,
            所以丢包的概率为$1 - \sum_{i = 0}^{b}e^{-\lambda}\frac{\lambda^i}{i!}$。
    \section*{6.}
        \subsection*{(a)}
            对于$n = 5$,凯尔特人队获胜的概率为$p^3 + C_3^1p^3(1 - p) + C_4^2p^3(1 - p)^2 = 10p^3 - 15p^4 + 6p^5$;

            对于$n = 3$,凯尔特人队获胜的概率为$p^2 + C_2^1p^2(1 - p) = 3p^2 - 2p^3$

            所以当$10p^3 - 15p^4 + 6p^5 > 3p^2 - 2p^3$,时,$n = 5$比$n = 3$合算。
            将所有式子移到左侧,消去$p^2(p > 0)$并求导得到$18p^2 - 30p + 12$,不难发现当$p = \frac{1}{2}$或$p = 1$时,原式为0,
            且该函数在$[0.5, 1]$上先增后减,所以当$p \in (0.5, 1)$时$n = 5$比$n = 3$合算。
        \subsection*{(b)}
            不难得到$P_N(n = 2k + 1) = p^{k + 1} + C_{k + 1}^1P^{k+1}(1 - p) + \dots + C_{2k}^kp^{k + 1}(1 - p)^k$,
            并且对于不同的k,$P_N(0) = 0, P_N(1) = 1, P_N(\frac{1}{2}) = \frac{1}{2}$恒成立(0和1十分显然,对于0.5的证明我们额外放在(c)部分讲解)。

            不妨做差$P_N(2k + 3) - P_N(2k + 1)$并对这个式子求导,不难验证导数在$\frac{1}{2}$处大于0。
            由罗尔定理导函数在$[0.5, 1]$之间一定存在值为0的点,且仅有一个(这是一个$2k + 1$重多项式,其中0为k重根,1为k重根,$\frac{1}{2}$为1重根)。
            所以导函数在区间上先增后减,原来的差值函数保证大于0,故$p \in (0.5, 1)$。
        \subsection*{(c)}
            原题不存在,这部分是(b)中$P_N(n)$在$p = \frac{1}{2}$为常数的证明。

            因为$P_N(n = 2k + 1) = p^{k + 1} + C_{k + 1}^1p^{k+1}(1 - p) + \dots + C_{2k}^kp^{k + 1}(1 - p)^k$,
            将$\frac{1}{2}$带入后$P_N(2k + 1) = (\frac{1}{2})^{k + 1} + C_k^1(\frac{1}{2})^{k + 2} + \dots + C_{2k}^k(\frac{1}{2})^{2k + 1}$。
            
            当$k = 0$时,原式显然成立。假定$k = m$时原式也成立,则$k = m + 1$时,
            我们不妨计算$2^{k+2}P_N(2m + 3), 2^{k + 1}P_N(2m + 1)$,将两个式子逐项做差,得到$2^{k + 1}P_N(2m + 1)$,
            则$2^{k+2}P_N(2m + 3) = 2^{k + 2}P_N(2m + 1)$,故原式也为$\frac{1}{2}$,归纳完毕。

            证毕。
\end{document}