\documentclass[UTF8]{article}
\usepackage{ctex}
\begin{document}
    \section*{5.}
        设挑选一个学生,这个学生是天才为事件A,这个学生喜欢巧克力为事件B,那么:

        $P(A) = 0.6, P(B) = 0.7, P(A \cap B) = 0.4$

        所以$P(\bar A \cap \bar B) = 1 - P(A) - P(B) + P(A \cap B) = 0.1$。

        这题是12题容斥原理的应用。
    \section*{6.}
        因为偶数的概率是奇数的两倍,又因为$P(even) + P(odd) = 1$,所以$P(even) = \frac{2}{3}, P(odd) = \frac{1}{3}$。

        因为不同偶数/奇数面出现的概率相同,于是:

        $$\begin{array}{rcl}
            P(1) &= \frac{1}{9} \\
            P(2) &= \frac{2}{9} \\ 
            P(3) &= \frac{1}{9} \\ 
            P(4) &= \frac{2}{9} \\ 
            P(5) &= \frac{1}{9} \\
            P(6) &= \frac{2}{9}
        \end{array}$$

        所以点数小于4的概率为$P(1) + P(2) + P(3) = \frac{4}{9}$。
    \section*{7.}
        样本空间为:第一次时停止,第二次时停止,第三次时停止……
    \section*{8.}
        不妨设第n局获胜的概率为$P_n$,那么最终获胜的概率:
        
        $P = P_1(1 - P_2)P_3 + P_1P_2 + (1 - P_1)P_2P_3$

        化简上式得到$P = P_1P_3 + P_2(P_1 + P_3 - 2P_1P_3)$。

        对于函数$f(x, y) = x + y - 2xy$在x和y都属于[0, 1]区间内时,$x \geq x^2, y \geq y^2, (x - y)^2 \geq 0$,所以$x + y - 2xy \geq x^2 + y^2 -2xy = (x - y)^2 \geq 0$

        所以对于$(P_1 + P_3 - 2P_1P_3)$,这个值永远非负,故$P_2$最大时整个式子最大,且与$P_1,P_3$的顺序无关。

        证毕。
    \section*{9.}
        \subsection*{(a)}
            $P(A) = P(A \cap \Omega) = P(A \cap \bigcup_{i = 1}^nS_i) = P(\bigcup_{i = 1}^n(A \cap S_i))$

            由于$\{S_1, S_2, ..., S_n\}$互不相容,故$P(A) = \sum_{i = 1}^nP(A \cap S_i)$

            证毕。
        \subsection*{(b)}
            不难发现,$B^c \cap C^c, B \cap C^c, B^c \cap C, B \cap C$互不相容

            所以$P(A) = P(A \cap B^c \cap C^c) + P(A \cap B \cap C^c) + P(A \cap B^c \cap C) + P(A \cap B \cap C)$

            由容斥原理,$P(A) = P(A \cap B^c \cap C^c) + P(A \cap B) + P(A \cap C) - P(A \cap B \cap C)$

            证毕。
    \section*{10.}
        $P(A) + P(B) - 2P(A \cap B) = P(A \cup B) - P(A \cap B) = P((A \cap B^c) \cup (A^c \cap B))$

        证毕。
    \section*{11.}
        \subsection*{(a)}
            原文给出的等式应为$P(A \cup B) = P(A) + P(B) - P(A \cap B)$,似乎存在笔误。
        \subsection*{(b)}
            略。
    \section*{12.}
        \subsection*{(a)}
            略。
        \subsection*{(b)}
            当n = 3时,由(a)可知等式成立。

            假定n = m时等式依然成立,那么当n = m + 1时:

            $P(\bigcup_{k = 1}^{m + 1}) = P(A_{m + 1} \cup \bigcup_{k = 1}^m A_k) = 
            P(A_{m + 1}) + \sum_{i \in S_1}P(A_i) 
            - \sum_{(i_1, i_2) \in S_2}P(A_{i_1} \cap A_{i_2}) + ... + (-1)^{n - 1}P(\bigcap_{k = 1}^{n}A_k) -
            P(A_{m + 1} \cap \bigcup_{k = 1}^m A_k)$

            整理上式可得:

            $P(\bigcup_{k = 1}^{m + 1}) = \sum_{i \in S_1}P(A_i) - \sum_{(i_1, i_2) \in S_2}P(A_{i_1} \cap A_{i_2}) + ... + (-1)^nP(\bigcap_{k = 1}^{m + 1}A_k)$

            归纳完毕。
    \section*{13.}
        略。
\end{document}