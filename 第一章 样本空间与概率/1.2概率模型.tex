\documentclass[UTF8]{article}
\usepackage{ctex}
\begin{document}
    \section*{5.}
        设挑选一个学生,这个学生是天才为事件A,这个学生喜欢巧克力为事件B,那么:

        $P(A) = 0.6, P(B) = 0.7, P(A \cap B) = 0.4$

        所以$P(\bar A \cap \bar B) = 1 - P(A) - P(B) + P(A \cap B) = 0.1$。

        这题是12题容斥原理的应用。
    \section*{6.}
        因为偶数的概率是奇数的两倍,又因为$P(even) + P(odd) = 1$,所以$P(even) = \frac{2}{3}, P(odd) = \frac{1}{3}$。

        因为不同偶数/奇数面出现的概率相同,于是:

        $$\begin{array}{rcl}
            P(1) &= \frac{1}{9} \\
            P(2) &= \frac{2}{9} \\ 
            P(3) &= \frac{1}{9} \\ 
            P(4) &= \frac{2}{9} \\ 
            P(5) &= \frac{1}{9} \\
            P(6) &= \frac{2}{9}
        \end{array}$$

        所以点数小于4的概率为$P(1) + P(2) + P(3) = \frac{4}{9}$。
    \section*{7.}
        样本空间为:第一次时停止,第二次时停止,第三次时停止……
    \section*{8.}
        不妨设第n局获胜的概率为$P_n$,那么最终获胜的概率:
        
        $P = P_1(1 - P_2)P_3 + P_1P_2 + (1 - P_1)P_2P_3$

        化简上式得到$P = P_1P_3 + P_2(P_1 + P_3 - 2P_1P_3)$。

        对于函数$f(x, y) = x + y - 2xy$在x和y都属于[0, 1]区间内时,假设$x < y$(反之亦然),有$f(x, y) \geq 2x - 2x^2 \geq 0$,

        所以对于$(P_1 + P_3 - 2P_1P_3)$,这个值永远非负,故$P_2$最大时整个式子最大,且与$P_1,P_3$的顺序无关。

        证毕。
\end{document}