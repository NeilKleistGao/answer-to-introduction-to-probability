\documentclass[UTF8]{report}
\usepackage{ctex}

\begin{document}
    \section*{30.}
        如果两头猎犬均选择了正确方向,概率为$p^2$;
        如果只有一头选择了正确方向,概率为$2p(1 - p)$,此时随机选择到正确方向的概率是$p(1 - p)$,
        故在这个策略下选择正确方向的概率为$p^2 + p - p^2 = p$,并不会比只让一条猎犬选择更优。
    \section*{31.}
        \subsection*{(a)}
            由于不同信号之间独立,第k个信号正确传输的概率为$p(1 - \epsilon_0) + (1 - p)(1 - \epsilon_1)$
        \subsection*{(b)}
            1011一共含有三个1和一个0,且彼此独立,所以正确传输的概率为$(1 - \epsilon_0)(1 - \epsilon_1)^3$
        \subsection*{(c)}
            只有以下的情况数据可以被正常传输:

            \begin{itemize}
                \item 3个0均没有错误,概率为$(1 - \epsilon_0)^3$
                \item 只有一个0发生错误,概率为$C^1_3\epsilon_0(1 - \epsilon_0)^2$
            \end{itemize}

            综上,0被正确传输的概率为$(1 - \epsilon_0)^3 + 3\epsilon_0(1 - \epsilon_0)^2$
        \subsection*{(d)}
            将(c)中的式子变形为$1^3 - 3\epsilon_0^2(1 - \epsilon_0) - \epsilon_0^3 = 1 - 3\epsilon_0^2 + 2\epsilon_0^3$

            对上式求导得到$-6\epsilon_0 + 6\epsilon_0^2$,由于$0 \leq \epsilon_0 \leq 1$,所以函数在$\epsilon_0 = 0$时取到最大值。
            即:整个传输过程完全不要出错是坠吼的。
        \subsection*{(e)}
            由贝叶斯准则,对方发出的数据为0的概率为$\frac{P(0)}{P(1) + P(0)} = \frac{\epsilon_0^2}{\epsilon_1 + \epsilon_0^2}$
    \section*{32.}
        \begin{itemize}
            \item 因为各次生育是独立的,所以国王的性别并不会影响他的兄弟姐妹,而生男的概率为$\frac{1}{2}$,故国王有一个兄弟的概率为$\frac{1}{2}$
            \item 国王的母亲一共有两个孩子,并且两个都是男性,所以国王有一个兄弟的概率为$\frac{1}{4}$
        \end{itemize}
    \section*{33.}
        投一次硬币已经没办法解决这个问题了,那我们考虑投两次。投两次硬币的全部概率为$1^2 = (1 - p + p)^2 = (1 - p)^2 + 2p(1 - p) + p^2$。注意到里面有一个2,
        那么我们不妨投两次硬币,如果两次结果相同,则重新开始;剩余的两种情况每个人获胜的概率均为$\frac{1}{2}$。
    \section*{34.}
        \begin{itemize}
            \item 第一个子系统有效的概率为$p$
            \item 第二个子系统有效的概率为$p + 3p^2 - 5P^3 + 2p^4$
            \item 第三个子系统有效的概率为$2p - p^2$
        \end{itemize}
        由于三个子系统独立,整个系统的有效概率为$p(p + 3p^2 - 5p^3 + 2p^4)(2p - p^2) = 2p^3 + 5p^4 -13p^5 + 9p^6 -2p^7$
    \section*{35.}
        由独立性:
        $\sum_{i = k}^nC^i_np^i(1 - p)^{n - i}$
    \section*{36.}
        \subsection*{(a)}
            只有所有电厂均中断的时候全市才会停电,故由独立性可得:
            $\Pi_{i = 1}^np_i$
        \subsection*{(b)}
            如果所有电厂中断,或者只有不到三个电厂在供电,全市会处在停电状态。
            设$P(i), P(i, j)$分别为仅第i个电厂工作的概率和仅第i,j个电厂在工作的概率,
            即$P(i) = \frac{(1 - p_i)\Pi_{j = 1}^np_j}{p_i}, P(i, j) = \frac{(1 - p_i)(1 - p_j)\Pi_{k = 1}^np_k}{p_ip_j}$,
            全市停电的概率为$\Pi_{i = 1}^n + \sum_{i = 1}^nP(i) + \sum_{1 \leq i < j \leq n}P(i, j)$。
    \section*{37.}
        $\sum_{i = 0}^{n_1}\sum_{j = 0}^{n_2}\left[r_1i + r_2j > c\right]C_{n_1}^ip_1^i(1 - p_1)^{n_1 - i}C_{n_2}^jp_2^j(1 - p_2)^{n_2 - j}$
    \section*{38.}
        考虑泰里思领先的概率$p_T$:只有在剩下的比赛中得到至少6分,才能保证泰里思领先。
        所以$p_T = \sum_{i = 6}^8C_8^ip^i(1 - p)^{8 - i}$。

        同理对于温迪来说,$P_W = \sum_{i = 4}^8C_8^i(1 - p)^ip^{8 - i} = \sum_{i = 0}^4C_8^ip^i(1 - p)^{8 - i}$。

        所以泰里思可以分到的钱为$\frac{10p_T}{p_T + p_W} = \frac{10p_T}{1 - p^5(1 - p)^3}$
    \section*{39.}
        设这天是好天气的概率为$p_w$,若当天每个学生的出勤概率为$t$,
        则当天出勤人数不少于k的概率$P_k(t) = \sum_{i = k}^nC_n^it^i(1 - t)^{n - i}$,
        那么教授可以讲课的概率为$p_wP_k(p_g) + (1 - p_w)P_k(p_b)$。
    \section*{40.}
        当$n = 0$时,$p_n = 1$。

        当$n > 0$时,$p_n = p(1 - p_{n - 1}) + p_{n - 1}(1 - p)$。

        所以$p_n = p_{n - 1}(1 - 2p) + p$。令$p_n' = p_n - \frac{1}{2}$,则$p_n' = (1 - 2p)p_{n - 1}'$,
        由等比数列通项公式:$p_n' = \frac{(1 - 2p)^n}{2}$,所以$p_n = \frac{1 + (1 - 2p)^n}{2}$。
    \section*{41.}
        假设第一个人转出来的数字为$p$,则在某一回合内,这个人被淘汰的概率为$p$。
        所以第一个人在第n个回合被淘汰的概率等于前n-1回合没被淘汰的概率乘上这个回合被淘汰的概率,
        即$P(N = n) = (1 - p)^{n - 1}p$。
    \section*{42.}
        略。这个问题在第七章马尔可夫链中还会再遇到。
    \section*{43.}
        略。
    \section*{44.}
        \subsection*{(a)}
            略。
        \subsection*{(b)}
            由(a)可知:若A和B独立,则$A$和$B^c$独立,
            即$P(A \cap B^c) = P(A)P(B^c)$。

            对于事件$B^c$,由全概率公式:
            $P(B^c) = P(A \cap B^c) + P(A^c \cap B^c) = P(A)P(B^c) + P(A^c \cap B^c)$,
            所以$P(A^c \cap B^c) = (1 - P(A))P(B^c) = P(A^c)P(B^c)$,所以$A^c$和$B^c$独立。

            证毕。
    \section*{45.}
        略。
    \section*{46.}
        略。
    \section*{47.}
        略。
    \section*{48.}
        略。
\end{document}