\documentclass[UTF8]{report}
\usepackage{ctex}

\begin{document}
    \section*{49.}
        第一次投掷:

        \begin{table}[!htbp]
            \begin{tabular}{lllllll}
                点数 & 1 & 2 & 3 & 4 & 5 & 6 \\
                概率 & $\frac{1}{6}$ & $\frac{1}{6}$ & $\frac{1}{6}$ &
                $\frac{1}{6}$ & $\frac{1}{6}$ & $\frac{1}{6}$
            \end{tabular}
        \end{table}

        第二次投掷:

        \begin{table}[!htbp]
            \begin{tabular}{llllllllllll}
                点数和 & 2 & 3 & 4 & 5 & 6 & 7 & 8 & 9 & 10 & 11 & 12 \\
                概率 & $\frac{1}{36}$ & $\frac{2}{36}$ & $\frac{3}{36}$ &
                $\frac{4}{36}$ & $\frac{5}{36}$ & $\frac{6}{36}$ &
                $\frac{5}{36}$ & $\frac{4}{36}$ & $\frac{3}{36}$ &
                $\frac{2}{36}$ & $\frac{1}{36}$
            \end{tabular}
        \end{table}

        第三次投掷:

        \begin{table}[!htbp]
            \begin{tabular}{lllllllllllllllll}
                点数和 & 3 & 4 & 5 & 6 & 7 & 8 & 9 & 10 & 11 & 12 & 13 & 14 & 15 & 16 & 17 & 18 \\
                概率 & $\frac{1}{216}$ & $\frac{3}{216}$ & $\frac{6}{216}$ &
                $\frac{10}{216}$ & $\frac{15}{216}$ & $\frac{21}{36}$ &
                $\frac{25}{216}$ & $\frac{27}{216}$ & $\frac{27}{216}$ &
                $\frac{25}{216}$ & $\frac{21}{216}$ & $\frac{15}{216}$ &
                $\frac{10}{216}$ & $\frac{6}{216}$ & $\frac{3}{216}$ &
                $\frac{1}{216}$
            \end{tabular}
        \end{table}

        从上表来看和为11的概率要大于和为12。
    \section*{50.}
        如果$n > 365$,那么根据鸽巢原理,一定有两个人的生日在同一天,所以这个概率是0。

        如果$1 < n \leq 365$,那么没有任何两个人在同一天生日的概率为$\frac{C_{365}^n}{365^n}$(每个人选其中的某一天作为自己的生日,可选的所有可能是$365^n$种)。
    \section*{51.}
        \subsection*{(a)}
            样本空间:抽走两个红球(设为事件A),抽走两个白球(设为事件B),抽走一个红球和一个白球(设为事件C)。

            基于离散均匀分布律的计数方法,我们有:
            $$P(A) = \frac{C_m^2}{C_{m + n}^2}, P(B) = \frac{C_n^2}{C_{m + n}^2},
             P(C) = \frac{C_m^1C_n^1}{C_{m + n}^2}$$

            设第一次抽到红球为事件D,第一次抽到白球为事件E,基于乘积规则,我们也可以得到:
            $$P(D) = \frac{m}{m + n}, P(E) = \frac{n}{m + n}$$

            那么:
            $$P(A) = \frac{m - 1}{m + n - 1}P(D) = \frac{C_m^2}{C_{m + n}^2} $$
            $$P(B) = \frac{n - 1}{m + n - 1}P(E) = \frac{C_n^2}{C_{m + n}^2} $$
            $$P(C) = \frac{m}{m + n - 1}P(E) + \frac{n}{m + n - 1}P(D) = \frac{C_m^1C_n^1}{C_{m + n}^2}$$
        \subsection*{(b)}
            \begin{itemize}
                \item 出现1的时候,全是红球的概率为$\frac{C_m^1}{C_{m + n}^1}$
                \item 出现2的时候,全是红球的概率为$\frac{C_m^2}{C_{m + n}^2}$
                \item 出现3的时候,全是红球的概率为$\frac{C_m^3}{C_{m + n}^3}$
            \end{itemize}

            由全概率公式,取出的球全是红色的概率为$\frac{1}{3}(\frac{C_m^1}{C_{m + n}^1} + \frac{C_m^2}{C_{m + n}^2} + \frac{C_m^3}{C_{m + n}^3})$。

    \section*{52.}
        第13张牌正好是老K,意味着前12张牌均不能是老K,故概率为$\frac{C_{48}^{12}}{C_{52}^{12}}\frac{C_4^1}{C_{40}^{1}}$。
    \section*{53.}
        我们把90个学生排成一排,前30个学生在一个班,中间30个学生在一个班,最后30个学生在一个班。这样所有的可能有$90!$种。
        对于乔和简,他们两个人可以先确定在三个班中的某一个,然后在这个班的30个位置中任意占据两个,其他人再随机排列。
        这样满足条件的排列数是$C_3^1C_{30}^298!$。所以分到一个班的概率为$\frac{C_3^1C_{30}^298!}{90!}$。
    \section*{54.}
        \subsection*{(a)}
            $\frac{20!}{10!10!} = 184756‬$
        \subsection*{(b)}
            不难发现只有两种排列方式满足要求(“美国车,外国车,美国车,\dots,外国车”和“外国车,美国车,外国车,\dots,美国车”)。
            所以这个概率为$\frac{2}{184756}$。
    \section*{55.}
        总共可以摆放的方案一共是$C_{64}^8$种。考虑合法的情况,每一个车一定单独占据一行,第一行的车有8种选择,第二行的车只有7种选择……
        所以合法的概率为$\frac{8!}{C_{64}^8}$。
    \section*{56.}
        \subsection*{(a)}
            $C_8^4C_{10}^3 = 1680$
        \subsection*{(b)}
            \begin{itemize}
                \item 假定所选的所有高水平课程均在$H_1, \dots, H_5$中,那么所有的可能有$C_7^3C_5^3 = 350$种;
                \item 假定所选的所有高水平课程均在$H_6, \dots, H_10$中,那么所有的可能有$C_6^2C_5^3 = 150$种;
                \item 假设二者均有涉及,那么三门先修课程均需要开设,所有的可能有$C_5^1\sum_{i = 1}^2C_5^iC_5^{3 - i} = 500$种
            \end{itemize}

            所以一共有1000种。
    \section*{57.}
        将问题换一个方式陈述:将一个26个不同字母随机排列组成的字符串划分为6个部分,有多少种划分方式?
        

        26个字母的总排列数是$26!$,将长度为26的字符串划分为6个部分,只需要在其中的25个间隔处插入空格即可,且不能有连续的空格。
        所以最后的结果为$26!C_{25}^5$。
    \section*{58.}
        接下来的问题中我们假设一共有52张牌。
        \subsection*{(a)}
            $\frac{C_4^3C_{48}^4}{C_{52}^7}$
        \subsection*{(b)}
            $\frac{C_4^2C_{48}^5}{C_{52}^7}$
        \subsection*{(c)}
            $\frac{C_4^3C_{48}^4}{C_{52}^7} + \frac{C_4^2C_{48}^5}{C_{52}^7} - \frac{C_4^2C_4^3C_{44}^2}{C_{52}^7}$
    \section*{59.}
        $\frac{C_k^nC_{100 - k}^{m - n}}{C_{100}^m}$

        柠檬法案:柠檬法(Lemon Laws)是一种美国的消费者保护法,主要是在保障汽车买主的权益。柠檬法的名称起源于美国经济学家乔治·阿克罗夫(George A. Akerlof)所发表的一篇经济学论文,因为这缘故,对于出厂后有瑕疵问题的汽车,通常也会称呼其为柠檬车(Lemon Car)或直接就称为柠檬。
    \section*{60.}
        我们可以使用类似于53题的思路,得到$\frac{13^4 \times \frac{48!}{(4!)^{12}}}{\frac{52!}{(4!)^{13}}} \approx 0.105$
    \section*{61.}
        略。
    \section*{62.}
        略。
\end{document}