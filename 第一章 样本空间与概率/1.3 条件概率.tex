\documentclass[UTF8]{report}
\usepackage{ctex}
\begin{document}
    \section*{14.}
        \subsection*{(a)}
            仅有(1, 1), (2, 2), (3, 3), (4, 4), (5, 5), (6, 6)满足情况,故$P = \frac{1}{6}$
        \subsection*{(b)}
            若总和小于等于4,则有(1, 1), (1, 2), (1, 3), (2, 1), (2, 2), (3, 1)六种情况,其中一对的事件有两个,故$P = \frac{1}{3}$。
        \subsection*{(c)}
            设两个骰子都是6为事件A,恰有一个骰子为6为事件B,则$P(A) + P(B) = \frac{1}{36} + \frac{10}{36} = \frac{11}{36}$即为所求。
        \subsection*{(d)}
            不同的抛掷结果一共有30种,而至少一个骰子为6(也只能有一个骰子为6)有10种情况,所以$P = \frac{1}{3}$
    \section*{15.}
        在已知第一次为正面的情况下,两次均正面朝上的概率只取决于第二次抛掷的情况,所以此时的概率为$\frac{1}{2}$。

        而两次中至少有一次正面朝上的概率为$\frac{3}{4}$,在已知该前提下,两次正面的概率为$\frac{1}{3}$,所以爱丽丝的结论是正确的。

        假定硬币不对称,即$p \ne 1 - p$,那么第一种情况的概率变为$p$;而第二种情况下至少一次正面朝上的概率变为了$1 - (1 - p)^2 = 2p - p^2$,
        在该前提下两次正面朝上的概率就应该是$\frac{p}{2 - p}$。因为$p \leq 1$,所以仅当$p = 1$或$p = 0$,即正面始终朝上或朝下时,两个条件概率相等,
        其他条件下,前者的概率都要比后者大。

        可以将这个结论推广到任意n$(n \geq 2)$次游戏:将前n - 1次游戏抛得正面条件下得到n次正面的概率与至少n-1次得到正面条件下的概率相差,
        得到$\frac{(n - 1)p + (1 - n)p^2}{p + n - np}$,由于$p \geq p^2$,可以得到和上面一样的结论。
    \section*{16.}
        所有抛掷可以得到正面的概率为$\frac{1}{3} + \frac{1}{6} = \frac{1}{2}$,而选中正常硬币并抛掷得到正面的概率为$\frac{1}{3} \times \frac{1}{2} = \frac{1}{6}$,
        于是由条件概率可得$\frac{\frac{1}{6}}{\frac{1}{2}} = \frac{1}{3}$。

        也可以考虑所有抛掷的结果:一共是三正三反,而三个正面朝上的情况中只有其中一个是来自于正常硬币,结果也是$\frac{1}{3}$。
    \section*{17.}
        设这批产品被接受为事件A,则$P(A) = \frac{C_{96}^{5}}{C_{100}^5}$,那么被拒绝的概率就为$1 - P(A) \approx 0.19$
    \section*{18.}
        $P(A \cap B|B) = \frac{P(A \cap B \cap B)}{P(B)} = \frac{P(A \cap B)}{P(B)} = P(A|B)$

        证毕。
\end{document}