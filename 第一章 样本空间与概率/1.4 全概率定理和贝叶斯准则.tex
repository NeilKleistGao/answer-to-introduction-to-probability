\documentclass[UTF8]{report}
\usepackage{ctex}
\begin{document}
    \section*{19.}
        若$j \ne i$,则在第i个抽屉内找但没有找到的概率为$1 - p_id_i$(要么不在这个抽屉,要么在但是没有找到),而报告在第j个抽屉的概率为$p_j$,
        故结果为$\frac{p_j}{1 - p_id_i}$。

        若$j = i$,在第i个抽屉内找但没有找到的概率同上,而报告恰好就在第i个抽屉并且没有被找到的概率为$P_i(1 - d_i)$,
        故结果为$\frac{p_i(1 - d_i)}{1 - p_id_i}$

        证毕。
    \section*{20.}
        \subsection*{(a)}
            \subsubsection*{(i)}
                进攻风格下,和局的概率为0,所以我们只需要考虑两种情况:
                前两局鲍里斯均获胜,前两局平局,第三局鲍里斯获胜。

                所以获胜概率为:$p_w^2 + p_w^2(1 - p_w)$。
            \subsubsection*{(ii)}
                保守风格下,获胜的概率为0,所以获胜情况只有一种:
                前两局和局,第三局获胜。

                所以获胜概率为:$p_d^2p_w$。
            \subsubsection*{(iii)}
                由全概率定理,整理出所有获胜情况:

                \begin{itemize}
                    \item 第一局获胜,第二局平局
                    \item 第一局获胜,第二局失败,第三局获胜
                    \item 第一局失败,第二局获胜,第三局获胜
                \end{itemize}

                所以获胜概率为:$p_wp_d + p_w^2(1 - p_d) + p_w^2(1 - p_w)$
        \subsection*{(b)}
            由(a)(iii)可知,策略三的获胜概率为$p_w(p_d + 2p_w - p_w^2 - p_wp_d)$。
            
            不难发现若$p_w = 0.49$时,$\exists p_d \approx 0.56$,使得上式可以大于0.5。
    \section*{21.}
        \begin{itemize}
            \item 若在第一回合获胜,概率为$\frac{m}{m + n}$
            \item 若在第二回合获胜,概率为$\frac{n}{m + n}\frac{n - 1}{m + n - 1}\frac{m}{m + n - 2}$
            \item 若在第三回合获胜,概率为$\frac{n}{m + n}\frac{n - 1}{m + n - 1}\frac{n - 2}{m + n - 2}\frac{n - 3}{m + n - 3}\frac{m}{m + n - 4}$
            \item ...
        \end{itemize}
        
        从上面的式子并不容易看出递推式来,但如果稍作一些改动:
        \begin{itemize}
            \item 若第一回合前无法决出胜负,概率为$1$
            \item 若第二回合前无法决出胜负,概率为$\frac{n}{m + n}\frac{n - 1}{m + n - 1}$
            \item 若第三回合前无法决出胜负,概率为$\frac{n}{m + n}\frac{n - 1}{m + n - 1}\frac{n - 2}{m + n - 2}\frac{n - 3}{m + n - 3}$
            \item ...
        \end{itemize}
        
        设第i回合前无法决出胜负的概率为$p_i$,那么我们可以得到$p_1 = 1$,递推式$p_i = p_{i - 1}\frac{n - 2i + 4}{m + n - 2i + 4}\frac{n - 2i + 3}{m + n -2i + 3}$

        所以获胜概率为:$\sum_{i = 1}^{1 + \lfloor \frac{n}{2} \rfloor}p_i \frac{m}{m + n - 2i + 2}$
    
    \section*{22.}
        当$k = 1$时,取出白球的概率为$\frac{m}{m + n}$显然成立。

        不妨假设当$k = p$时,在第k个罐子里取出白球的概率依然是$\frac{m}{m + n}$,
        则对于$k = p + 1$时,由全概率定理,
        在第k个罐子里取出白球的概率为
        
        $\frac{m}{m + n}\frac{m + 1}{m + n + 1} + \frac{n}{m + n}\frac{n}{m + n + 1}$

        整理后得到$\frac{m}{m + n}$,归纳完毕。

        证毕。
    \section*{23.}
        不妨设罐子里的球的个数为n,分两种情况讨论:
        \begin{itemize}
            \item 第一次交换两个球,第二次再交换回来,第三第四次亦如此,这样做的概率为$(\frac{1}{n})^4$
            \item 第一次第二次一共交换了两组球,第三第四次再交换回来,这样做的概率为$\frac{4(n - 1)^2}{n^6}$
        \end{itemize}
        总的概率为二者之和。
    \section*{24.}
        $\frac{2}{3}$是先验概率,而$\frac{1}{2}$是后验概率。
    \section*{25.}
        不妨设两个信封内的钱数分别为S和L,且$L > S$。

        对于S,选中的概率为$\frac{1}{2}$,且只有抛硬币的次数不少于S次时才会换信封。
        若第一次选中了该信封,则最后得到的钱数为L的概率为

        $\frac{1}{2}(1 - \sum_{i = 1}^{S - 1}(\frac{1}{2})^i) = (\frac{1}{2})^S$

        对于L,选中的概率也是$\frac{1}{2}$,且只有抛硬币的次数小于L次时才会选择不换信封。
        若第一次选中了该信封,则最后得到的钱数为L的概率为

        $\frac{1}{2}\sum_{i = 1}^{L - 1}(\frac{1}{2})^i = \frac{1}{2} - (\frac{1}{2})^L$

        所以最后的概率为$\frac{1}{2} - (\frac{1}{2})^L + (\frac{1}{2})^S$。因为$L > S$,所以这个概率大于$\frac{1}{2}$。
    \section*{26.}
        \subsection*{(a)}
            由于A和B独立,由条件概率$P(A|B) = \frac{p(1 - q)}{1 -q} = p$。
        \subsection*{(b)}
            $P(C) = q(p + \frac{1 - p}{2}) = \frac{q(1 + p)}{2}$

            由贝叶斯准则:$P(A|C) = \frac{P(A)P(C|A)}{P(C)} = \frac{2p}{p + 1}$
    \section*{27.}
        \begin{itemize}
            \item 当鲍勃的所有硬币均正面朝上时,爱丽丝无论怎么抛都是失败,一共有$2^n$种情况
            \item 当鲍勃有一枚硬币朝下时,对于鲍勃来说有$C^1_{n+1}$种可能,而对于爱丽丝来说,除了全正以外都是鲍勃获胜,故有$2^n - C^0_n$种情况
            \item 当鲍勃有两枚硬币朝下时,对于鲍勃来说有$C^2_{n+1}$种可能,对于爱丽丝来说,有$2^n - C^0_n - C^1_n$种情况
            \item \dots
        \end{itemize}
        综上,所有鲍勃获胜的情况的可能数一共是
        $2^n + C_{n + 1}^1(2^n - C_n^0) + C_{n + 1}^2(2^n - C_n^0 - C_n^1) + \dots + C_{n + 1}^n(2^n - C_n^0 - \dots - C_n^{n - 1})$

        上式化简后得到$2^{2n + 1} - 2^n - \sum_{i = 1}^nC_{n + 1}^i(\sum_{j = 0}^{i - 1}C_n^j)$。

        注意到$2^{2n + 1} = 2^{n + 1} \times 2^n = (\sum_{i = 0}^{n + 1}C_{n + 1}^i)(\sum_{i = 0}^{n}C_n^i)$

        所以$(\sum_{i = 0}^{n + 1}C_{n + 1}^i)(\sum_{i = 1}^{n}C_n^i) = 2^{2n + 1} - 2^{n + 1}$

        因为$\sum_{i = 1}^{n}C_{n + 1}^i(\sum_{j = 0}^{i - 1}C_n^j) = \sum_{i = 1}^{n}C_{n + 1}^i(\sum_{j = i}^{n}C_n^j)$

        所以
        $2\sum_{i = 1}^{n}C_{n + 1}^i(\sum_{j = 0}^{i - 1}C_n^j) = \sum_{i = 1}^{n}C_{n + 1}^i(\sum_{j = 0}^{i - 1}C_n^j) + \sum_{i = 1}^{n}C_{n + 1}^i(\sum_{j = i}^{n}C_n^j)$

        上式就等于$(\sum_{i = 1}^{n}C_{n + 1}^i)(\sum_{i = 0}^{n}C_n^i)$,即$2^{2n + 1} - 2^{n + 1}$

        带回原式可知鲍勃获胜的情况总共有$2^{2n}$种,所有情况一共有$2^{2n + 1}$种,故获胜的概率为$\frac{1}{2}$。

        证毕。
    \section*{28.}
        略。
    \section*{29.}
        略。
\end{document}