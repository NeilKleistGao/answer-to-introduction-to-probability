\documentclass[UTF8]{article}
\usepackage{ctex}
\begin{document}
    \section{}
        令事件A为掷出偶数,即$A = \{2, 4, 6\}$
        
        令B表示点数大于3的事件,即$B = \{4, 5, 6\}$

        那么:

        $A^c = \{1, 3, 5\}, B^c = \{1, 2, 3\}$

        $A \cup B = \{2, 4, 5, 6\}, A \cap B = \{4, 6\}$

        所以$(A \cup B)^c = \{1, 3\} = A^c \cap B^c$
        
        $(A \cap B)^c = \{1, 2, 3, 5\} = A^c \cup B^c$
    \section{}
        \subsection{}
            设全集为S,由代数性质:
            
            $\begin{array}{lcr}
                A^c & = \\
            A^c \cap S & = \\
            A^c \cap (B \cup B^c) & = \\
            (A^c \cap B) \cup (A^c \cap B^c)
            \end{array}$

            对$B^c$的结论同理,证毕。

            也可以通过韦恩图进行证明:任意一个集合中的元素,要么属于$B$,要么属于$B^c$。上式就是全概率定理的特殊情况。
        \subsection{}
            设全集为S,由德摩根律:

            $\begin{array}{lcr}
                (A \cap B)^c & = \\
                A^c \cup B^c & = \\
                (A^c \cap S) \cup (B^c \cap S) & = \\
                (A^c \cap B) \cup  (A^c \cap B^c) \cup (A^c \cap B^c) \cup (A \cap B^c) & = \\
                (A^c \cap B) \cup  (A^c \cap B^c) \cup (A \cap B^c)
            \end{array}$

            证毕。

            也可以通过韦恩图进行证明:A交B的补集中,只有三种可能:属于A但不属于B、属于B但不属于A、既不属于A也不属于B,整理即可得到上式。
        
        \subsection{}
            令事件A为掷出偶数,即$A = \{2, 4, 6\}$
        
            令B表示点数大于3的事件,即$B = \{1, 2, 3\}$

            则$(A \cap B)^c = \{1, 3, 4, 5, 6\}$

            $(A^c \cap B) \cup  (A^c \cap B^c) \cup (A \cap B^c) = \{1, 3\} \cup \{5\} \cup \{4, 6\} = \{1, 3, 4, 5, 6\}$

            所以$(A \cap B)^c = (A^c \cap B) \cup  (A^c \cap B^c) \cup (A \cap B^c)$。
\end{document}